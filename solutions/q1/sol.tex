\chapter{Question 1}
Find the exact values of $\sin\left(\frac{7\pi}{12}\right)$ and
$\cos\left(\frac{7\pi}{12}\right)$. Do not use a calculator, and explain your
reasoning carefully.

$\frac{7\pi}{12}$ is a fraction, in radians it is close to the 12 o'clock
position on the unit circle. We know this because $\frac{\pi}{2}$ radians
\textbf{is} the 12 o'clock position. Because $\frac{7}{12}$ is a little larger
than $\frac{1}{2}$, we also know the position will be just past 12, closer to 11
o'clock (fig \ref{fig:q1sums}).

\newpage

\begin{figure}[!h]
\centering
\begin{tikzpicture}[scale=3,cap=round,>=latex]
  % draw the coordinates
  \draw[dashed,->] (-1.5cm,0cm) -- (1.5cm,0cm) node[right,fill=white] {$x$};
  \draw[dashed,->] (0cm,-1.5cm) -- (0cm,1.5cm) node[above,fill=white] {$y$};

  % draw the unit circle
  \draw[thick] (0cm,0cm) circle(1cm);

  \draw[neekiRed] (0cm,0cm) -- (60:1cm);
  \filldraw[neekiRed] (60:1cm) circle(0.4pt);
  \draw (60:0.6cm) node[fill=white] {$\frac{\pi}{3}$};

  \draw[neekiBlue] (0cm,0cm) -- (45:1cm);
  \filldraw[neekiBlue] (45:1cm) circle(0.4pt);
  \draw (45:0.6cm) node[fill=white] {$\frac{\pi}{4}$};

  \draw[black] (0cm,0cm) -- (105:1cm);
  \filldraw[black] (105:1cm) circle(0.4pt);
  \draw (105:0.6cm) node[fill=white] {$\frac{7\pi}{12}=\frac{\pi}{3}+\frac{\pi}{4}$};

%  \foreach \x in {0,30,...,360} {
%    % lines from center to point
%    \draw[gray] (0cm,0cm) -- (\x:1cm);
%    % dots at each point
%    \filldraw[black] (\x:1cm) circle(0.4pt);
%    % draw each angle in degrees
%    \draw (\x:0.6cm) node[fill=white] {$\x^\circ$};
%  }

  % draw the horizontal and vertical coordinates
  % the placement is better this way
  \draw (-1.25cm,0cm) node[above=1pt] {$(-1,0)$}
  (1.25cm,0cm)  node[above=1pt] {$(1,0)$}
  (0cm,-1.25cm) node[fill=white] {$(0,-1)$}
  (0cm,1.25cm)  node[fill=white] {$(0,1)$};
\end{tikzpicture}
\caption{$\frac{7\pi}{12}$ in relation to $\frac{\pi}{2}$}
\label{fig:q1sums}
\end{figure}

\noindent $\frac{7\pi}{12}$ is an angle that we do not have identities for. We
\textbf{do} have identities for $\frac{\pi}{3}$ and $\frac{\pi}{4}$, and that:
\begin{align}
  \frac{7\pi}{12} &= \frac{\pi}{3} + \frac{\pi}{4} \quad \text{such that} \\
  \sin\left(\frac{7\pi}{12}\right) &= \sin\left(\frac{\pi}{3} + \frac{\pi}{4}\right)
  \intertext{and}
  \cos\left(\frac{7\pi}{12}\right) &= \cos\left(\frac{\pi}{3} + \frac{\pi}{4}\right)
  \intertext{We have identities for $\sin\left(\frac{\pi}{3}\right)$, $\sin\left(\frac{\pi}{4}\right)$,
$\cos\left(\frac{\pi}{3}\right)$ and $\cos\left(\frac{\pi}{4}\right)$, and can
  apply sine and cosine addition rules:}
  \sin(\alpha \pm \beta) &= \sin(\alpha)\cos(\beta) \pm \sin(\beta)\cos(\alpha) \nonumber \\
  \cos(\alpha \pm \beta) &= \cos(\alpha)\cos(\beta) \mp \sin(\alpha)\sin(\beta) \nonumber
  \intertext{Let:}
  \alpha &= \frac{\pi}{3} \nonumber \\
  \beta  &= \frac{\pi}{4} \nonumber \\
  \sin\left(\frac{\pi}{3} + \frac{\pi}{4}\right)
    &= \sin\left(\frac{\pi}{3}\right)\cos\left(\frac{\pi}{4}\right) +
       \sin\left(\frac{\pi}{4}\right)\cos\left(\frac{\pi}{3}\right) \\
  \intertext{substitute our identity values:}
    \sin \alpha &= \sin\left(\frac{\pi}{3}\right) = \frac{\sqrt{3}}{2} \nonumber \\
    \sin \beta  &= \sin\left(\frac{\pi}{4}\right) = \frac{\sqrt{2}}{2} \nonumber \\
    \cos \alpha &= \cos\left(\frac{\pi}{3}\right) = \frac{1}{2} \nonumber \\
    \cos \beta  &= \cos\left(\frac{\pi}{4}\right) = \frac{\sqrt{2}}{2} \nonumber \\
  \sin\left(\frac{7\pi}{12}\right) &= \sin\left(\frac{\pi}{3} + \frac{\pi}{4}\right)
    = \frac{\sqrt{3}}{2}\cdot\frac{\sqrt{2}}{2} +
      \frac{\sqrt{2}}{2}\cdot\frac{1}{2} \\
    &= \frac{\sqrt{3}}{2}\cdot\frac{1}{\sqrt{2}} +
    \frac{1}{\sqrt{2}}\cdot\frac{1}{2} \\
    &= \frac{\sqrt{3}}{2\sqrt{2}} + \frac{1}{2\sqrt{2}} \\
    &= \frac{\sqrt{3} +1}{2\sqrt{2}} \\
  \cos\left(\frac{7\pi}{12}\right) &= \cos\left(\frac{\pi}{3}\right) \cdot \cos\left(\frac{\pi}{4}\right)
    - \sin\left(\frac{\pi}{3}\right) \cdot \sin\left(\frac{\pi}{4}\right) \\
    &= \frac{1}{2} \cdot \frac{\sqrt{2}}{2} - \frac{\sqrt{3}}{2} \cdot \frac{\sqrt{2}}{2} \\
    &= \frac{1}{2} \cdot \frac{1}{\sqrt{2}} - \frac{\sqrt{3}}{2} \cdot \frac{1}{\sqrt{2}} \\
    &= \frac{1 - \sqrt{3}}{2\sqrt{2}}
\end{align}

\noindent Finally, we need to consider the quadrant. $2^{nd}$ quadrant means
sine is positive and cosine is negative. Hence: \\
\\
\Large{$\sin\left(\frac{7\pi}{12}\right) = \frac{\sqrt{3} +1}{2\sqrt{2}}$ } and \\
\Large{$\cos\left(\frac{7\pi}{12}\right) = - \frac{1 - \sqrt{3}}{2\sqrt{2}}$} \qedbitches \\
